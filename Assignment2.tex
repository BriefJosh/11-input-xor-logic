\documentclass{article}  

\usepackage{fancyhdr}  
\usepackage{extramarks}  
\usepackage{amsmath}  
\usepackage{amsthm}  
\usepackage{amsfonts}  
\usepackage{tikz}  
\usepackage[plain]{algorithm}  
\usepackage{algpseudocode}  
\usepackage{graphicx}  
\usepackage{hyperref}  
\usepackage{circuitikz}  
 
\ctikzset{logic ports=ieee, logic ports/scale=0.6}

% Basic Document Settings  
\topmargin=-0.45in  
\evensidemargin=0in  
\oddsidemargin=0in  
\textwidth=6.5in  
\textheight=9.0in  
\headsep=0.25in  
\linespread{1.1}  

\pagestyle{fancy}  
\lhead{Josh Christian}  
\rhead{Github : BriefJosh}  
\cfoot{\thepage}  

\renewcommand\headrulewidth{0.4pt}  
\renewcommand\footrulewidth{0.4pt}  
\setlength\parindent{0pt}  

\usetikzlibrary{automata,positioning} 


\begin{document}  

\section*{Cascaded vs Parallel XOR Logic}

\textbf{Cascaded XOR Logic:} In a cascaded XOR configuration, the output of one XOR gate is connected to the input of the next XOR gate in a sequential manner. This creates a chain of XOR gates where each gate processes the result of the previous gate along with a new input. This configuration is simple and easy to implement but can introduce a delay as the signal propagates through each gate in the sequence.

\textbf{Parallel XOR Logic:} In a parallel XOR configuration, multiple XOR gates operate simultaneously on different pairs of inputs. The outputs of these gates are then combined in subsequent layers of XOR gates until a single output is produced. This arrangement can be more complex to design but offers the advantage of reduced propagation delay, as multiple operations are performed concurrently.

\begin{center}  
    \textbf{Cascaded 11 Input XOR Logic}  
\end{center}  

\begin{circuitikz} \draw   
    % 1st XOR Model  
    (0,0) node[xor port](xor1) {}  
        (xor1.in 1) node[anchor=east] {$x_0$}  
        (xor1.in 2) node[anchor=east] {$x_1$}  

    %  2nd XOR Model    
    (1.5,-0.5) node[xor port] (xor2) {}  
    (xor1.out) |- (xor2.in 1)   
    (xor2.in 2) to (xor2.in 2 -| xor1.in 2)  
        node[anchor=east] {$x_2$}  

    % 3rd XOR Model  
    (3, -1) node[xor port] (xor3) {}  
    (xor2.out) |- (xor3.in 1)  
    (xor3.in 2) to (xor3.in 2 -| xor1.in 2)  
        node[anchor=east] {$x_3$}  

    % 4th XOR Model  
    (4.5, -1.5) node[xor port] (xor4) {}  
    (xor3.out) |- (xor4.in 1)  
    (xor4.in 2) to (xor4.in 2 -| xor1.in 2)  
        node[anchor=east] {$x_4$}  

    % 5th XOR Model  
    (6, -2) node[xor port] (xor5) {}  
    (xor4.out) |- (xor5.in 1)   
    (xor5.in 2) to (xor5.in 2 -| xor1.in 2)  
        node[anchor=east] {$x_5$}  
    
    % 6th XOR Model  
    (7.5, -2.5) node[xor port] (xor6) {}   
    (xor5.out) |- (xor6.in 1)  
    (xor6.in 2) to (xor6.in 2 -| xor1.in 2)  
        node[anchor=east] {$x_6$}  

    % 7th XOR Model  
    (9, -3) node[xor port] (xor7) {}  
    (xor6.out) |- (xor7.in 1)  
    (xor7.in 2) to (xor7.in 2 -| xor1.in 2)  
        node[anchor=east] {$x_7$}  

    % 8th XOR Model  
    (10.5, -3.5) node[xor port] (xor8) {}  
    (xor7.out) |- (xor8.in 1)  
    (xor8.in 2) to (xor8.in 2 -| xor1.in 2)  
        node[anchor=east] {$x_8$}  
    
    % 9th XOR Model  
    (12, -4) node[xor port] (xor9) {}  
    (xor8.out) |- (xor9.in 1)  
    (xor9.in 2) to (xor9.in 2 -| xor1.in 2)  
        node[anchor=east] {$x_9$}  
    

    % 10th XOR Model  
    (13.5, -4.5) node[xor port] (xor10) {}  
    (xor9.out) |- (xor10.in 1)  
    (xor10.in 2) to (xor10.in 2 -| xor1.in 2)  
        node[anchor=east] {$x_{10}$}  
        (xor10.out) node [anchor=west] {Output};  

\end{circuitikz}  

\begin{center}  
    \textbf{Parallel 11 Input XOR Logic}  
\end{center}  

\begin{circuitikz} \draw  
    % First Layer XOR Model [XOR 1 - 6]  
    (0,0) node [xor port] (xor1) {}  
    (0,-1) node [xor port] (xor2) {}  
    (0,-2) node [xor port] (xor3) {}  
    (0,-3) node [xor port] (xor4) {}  
    (0,-4) node [xor port] (xor5) {}  
    (xor1.in 1) node[anchor=east] {$x_0$}  
    (xor1.in 2) node[anchor=east] {$x_1$}  
    (xor2.in 1) node[anchor=east] {$x_2$}  
    (xor2.in 2) node[anchor=east] {$x_3$}  
    (xor3.in 1) node[anchor=east] {$x_4$}  
    (xor3.in 2) node[anchor=east] {$x_5$}  
    (xor4.in 1) node[anchor=east] {$x_6$}  
    (xor4.in 2) node[anchor=east] {$x_7$}  
    (xor5.in 1) node[anchor=east] {$x_8$}  
    (xor5.in 2) node[anchor=east] {$x_9$}  

    % Second Layer XOR Model [XOR 7 - 9]  
    (2,-0.5) node [xor port] (xor6) {}  
    (2, -2.5) node [xor port] (xor7) {}  
    (2, -4.5) node [xor port] (xor8) {}  
    (xor8.in 2) to (xor8.in 2 -| xor1.in 2)  
        node[anchor=east] {$x_{10}$}  

    % Connect First Layer to Second Layer  
    (xor1.out) |- (xor6.in 1)  
    (xor2.out) |- (xor6.in 2)  
    (xor3.out) |- (xor7.in 1)  
    (xor4.out) |- (xor7.in 2)  
    (xor5.out) |- (xor8.in 1)  

    % Third Layer XOR Model  
    (4, -1.5) node [xor port] (xor9) {}   

    % Connect Second Layer to Third Layer  
    (xor6.out) |- (xor9.in 1)  
    (xor7.out) |- (xor9.in 2)  

    % Fourth Layer XOR Model  
    (6, -2) node [xor port] (xor10) {}   

    % Connect Second Layer (XOR 8) and Third Layer (XOR 9) to Fourth Layer  
    (xor9.out) |- (xor10.in 1)
    (xor8.out) |- ++(1,1) |- (xor10.in 2)
    (xor10.out) node [anchor=west] {Output};
\end{circuitikz}  

\end{document}  